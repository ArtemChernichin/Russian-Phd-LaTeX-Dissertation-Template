\chapter*{Введение}							% Заголовок
\addcontentsline{toc}{chapter}{Введение}	% Добавляем его в оглавление

\newcommand{\actuality}{}
\newcommand{\progress}{}
\newcommand{\aim}{{\textbf\aimTXT}}
\newcommand{\tasks}{\textbf{\tasksTXT}}
\newcommand{\novelty}{\textbf{\noveltyTXT}}
\newcommand{\influence}{\textbf{\influenceTXT}}
\newcommand{\methods}{\textbf{\methodsTXT}}


Анализ доходностей финансовых стратегий c целью формирования портфеля включает в себя ряд специфичных особенностей.
У каждой стратегии есть автор, который видел результат своей работы, например коэффициент Шарпа. 
Стараясь получить прибыльную стратегию, автор неявно учитывает уникальные особенности доступного периода для валидации, но не видит дальнейшего поведения алгоритма. Возникает точка структурного сдвига. 
Ее учет - исследовательская задача, которая также будет затронута в работе.

Исследование проводилось внутри компании \textbf{Quantopian LLC}\footnote{\href{https://www.quantopian.com}{https://www.quantopian.com}}, которая предоставила доступ к данным и необходимым вычислительным мощностям. Quantopian -- хедж фонд, аггрегирующий и комбинирующий торговые стратегии для проведения операций на бирже.

{\aim} данной работы является разработка метода оптимизации портфеля торговых стратегий.

Для достижения поставленной цели было необходимо решить следующие {\tasks}:
\begin{enumerate}
  \item Проверить гипотезы необходимости учета корреляций доходностей
  \item Дополнить существующую математическую модель учитавая результаты проверки гипотез
  \item Написать модель на языке \texttt{python}
  \item Протестировать модель на реальных данных
\end{enumerate}

{\novelty}
\begin{enumerate}
  \item Впервые были применены байесовские методы для оптимизации портфеля торговых стратегий
\end{enumerate}

{\influence} данного исследования заключается в \todo{TODO}


 % Характеристика работы по структуре во введении и в автореферате не отличается (ГОСТ Р 7.0.11, пункты 5.3.1 и 9.2.1), потому её загружаем из одного и того же внешнего файла, предварительно задав форму выделения некоторым параметрам

%\textbf{Объем и структура работы.}

