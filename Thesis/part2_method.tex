\chapter{Теоретические основы моделирования динамики доходности торговых стратегий}
\section{Моделирование сдвигов}

\section{Моделирование волатильности}
\section{Моделирование динамики корреляций}
\section{Байесовский подход к моделированию динамики торговых стратегий}
\section{Структурная модель динамики доходностей торговых стратегий}



\begin{comment}
В мировой литературе хорошо раскрыта портфельная теория. \cite{markovitz1959} в своей работе показал, что можно эффективно составлять портфель с максимальным коэффициентом Шарпа. Веса портфеля однозначно определяются формулой
\begin{align}
w^\star = \frac{\Sigma^{-1}\mu}{\textbf{1}^\top\Sigma^{-1}\mu},
\end{align}
Где $\mu = \Eb{r}-r_f$ -- матожидание премии за риск активов, $\Sigma = Var(r)$ -- ковариационная матрица доходностей активов. 
Портфель, полученный таким образом, может быть использован для инвестирования с горизонтом в один период. Однако, остается открытым вопрос как считать $\E{r}$, $\Sigma$. Обычно это делается на основе последнего разумной длины среза во времени (окна) .

\cite{billio2003} поднимает вопрос методологии оценки $\E{r}$, $\Sigma$. Авторы заостряли внимание на том, что разный выбор окна значительно влияет на результат. Более того, выбор длины окна вовсе не очевиден. Для изучаемой проблемы данные для агрегирования ограничены, длину окна нельзя сделать произвольно большой из-за существования даты создания алгоритма и наоборот, маленькой, из-за ненадежности точечных оценок.

Один из путей решения проблемы короткого окна -- снизить число оцениваемых параметров ковариационной матрицы \citep{engle2012}. Для этого можно ввести предположение о структуре ковариационной матрицы.
\begin{enumerate}
	\item Ряды группируются в несколько кластеров
	\item В каждом кластере корреляция общая для всех пар рядов,  $\Sigma_c=(1-\alpha)\mathbb{I} + \alpha \textbf{1}\textbf{1}^\top$
\end{enumerate}
Различные модели были предложены и для многомерных рядов с непостоянными корреляциями и дисперсией остатков. Это позволит учесть критику \cite{billio2003} и построить единую модель для всего периода.
\end{comment}