\chapter{Описание проблемы}
\section{Торговые стратегии}
\textit{Торговая стратегия (алгоритм)} -- набор заранее заданных правил по проведению операций на бирже. 
Эти правила определяют какую позицию (короткую или длинную) занять по каждому инструменту с определенной периодичностью. Таким образом алгоритм формирует динамический портфель, стоимость которого можно посчитать в любой момент времени. Понятие доходности применимо для торговых стратегий.

Пусть $\;\mathcal{U}$ -- множество доступных для торговли инструментов, $\mathcal{H}_t$ -- информационное множество доступное в момент времени $t$, $\textbf{p}_t$ -- цены инструментов в момент $t$,  $\mathcal{A}$ - множество стратегий таких, что:
\begin{align}
	a \in \mathcal{A} \colon \mathcal{H}_t \rightarrow \textbf{w}_t &\in \mathbb{R}^{|\mathcal{U}|}\\ 
	\sum_{i=1}^{|\mathcal{U}|} w_{it} &= 1,
\end{align}
где $\{w_{it}\} = \textbf{w}_t$ - веса инструментов $\{u_i\} = \mathcal{U}$ соответственно. 
Изменение структуры портфеля происходит с комиссией. 
Размер комиссии зависит от старого и нового набора $\textbf{w}$, обозначим
\begin{align}
	\textit{C} \colon \textbf{w} &\times \textbf{w} \rightarrow \mathbb{R}_+\\
	\textit{C}(\textbf{w}, \textbf{w}') &= 0 \iff \textbf{w} = \textbf{w}'	
\end{align}
Таким образом стоимость нового портфеля $R_t$ выражается формулой
\begin{align}
	\textbf{w}_{t} &= a(\mathcal{H}_t)\\
	R_t &= \textbf{w}_{t}^\top\textbf{p}_t - \textit{C}(\textbf{w}_{t}, \textbf{w}_{t-1})
\end{align}
Инвестор надеется, что в будущем $R_{t+1} > R_t$. 
Комиссия играет важную роль в алгоритмической торговле. 
Слишком частые действия на рынке могут влечь за собой высокие транзакционные издержки и быть нецелесообразными.
\subsection{Частные случаи}
\paragraph{Buy \& Hold}
Заметим, что если $a(\mathcal{H}_t) = const$, то мы получим стратегию <<buy and hold>>, равносильную пассивному держанию портфеля активов. Её транзакционные издержки будут нулю.
\paragraph{Актив}
Если же $a(\mathcal{H}_t)=\textbf{w}$ имеет вид $(0, 0, \dots, \underset{i}{1}, \dots, 0)^\top$, то это будет соответствовать тривиальной длинной позиции по одному активу $i$.
\subsection{Отличия торговых стратегий от портфелей и активов}
Каждая торговая стратегия - творение человека, \textit{автора}. 
И человек, что ее создавал обладал информацией о результативности своей стратегии на исторических данных.
Стараясь сделать хорошую стратегию, автор неявно учитывает особенности периода, который доступен для бэктеста.
Бэктест -- анализ стратегии на исторических данных, включает в себя расчет статистических показателей: коэффициент Шарпа, среднюю доходность, риск.

Таким образом стратегия, которая себя может и показала хорошо на бэктесте, может быть совершенно непредсказуемой после него.
Момент создания стратегии -- структурный сдвиг, который необходимо учитывать, оценивая истинную доходность стратегии.


